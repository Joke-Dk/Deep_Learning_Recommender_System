% !TEX program = pdflatex
\documentclass{article}
\usepackage{listings}
\usepackage{tasks}
\usepackage{graphicx}
\usepackage{hyperref}

\pagenumbering{gobble}

\usepackage[                                                                       
 paper  = letterpaper,                                                            
 left   = 1.5in,                                                                 
 right  = 1.5in,                                                                 
 top    = 1.0in,                                                                  
 bottom = 1.0in,                                                                  
 ]{geometry}

\begin{document}

\title{Deep Learning $|$ Final Project Proposal}
\author{Michael Alvarino (maa2282), Richard Dewey (rld2126), Colby Wise (cjw2165)}
\date{\today}

\maketitle

\section{Project Review} We intend to build a best-in-class movie recommendation system that will build upon standard collaborative filtering techniques by incorporating additional contextual information and state-of-the-art neural network architectures. 

\paragraph{The Problem} The standard approach for movie recommendation systems has been to use a collaborative filtering approach. This approach relies on a matrix with users, items (movies) and ratings (as the elements of the matrix).  Matrix factorization is then used to recommend items (movies) to the user. These approaches suffer from sparsity which pushed recent research to use contextual data, such as reviews posted by users, in an attempt to improve the recommendation systems.

\paragraph{Data} We intend to use three data sources each of which are open source and have been used in previous studies. Those sources are: movielens, netlfix and imdb. The datasets have overlap in movies and ratings, but each dataset also has distinctive features, which makes them part of a useful complimentary set. 

\paragraph{Previous Work and References} The literature on recommendation systems is deep. We plan to focus on the following four papers which all offer new approaches using contextual data and novel approaches. 

\begin{itemize}
\item Cooperative Deep Learning. 1 NN for rating/1NN for content.Combined using matrix factorization
\subitem Zheng and Noroozi. "Joint Deep Modeling of Users and Items Using Reviews for Recommendation"
\item Collaborative Deep Learning. DL for learning content. CF for ratings. Jointly modeled. 
\subitem Wang and Wang. "Collaborative Deep Learning for Recommender Systems"
\item CNN integrated with matrix factorization
\subitem Kim, Donghyun. "Convolutional Matrix Factorization for Document Context-Aware Recommendation"
\item Deep and Wide NNs. Generalization and Memorization
\subitem Google Research paper
\end{itemize}


\paragraph{Evaluation Criteria} We plan to evaluate our model against the standard collaborative filtering techniques and against more recent approaches that incorporate additional data and novel techniques. Below is a list of our benchmarks. 

\begin{itemize}
\item Standard rating prediction model that only uses rating for collaborative filtering
\subitem Salakhutdinov, Ruslan and Mnih, Andriy. "Probabilistic Matrix Factorization" 
\item A state-of-the-art recommendation model that combines collaborative filtering (PMF) and topic modeling (LDA) to use both ratings and documents.
\subitem Wang, Chong and Blei, David. "Collaborative Topic Modeling for Recommending Scientific Articles"
\item A state-of-the-art recommendation model that combines collaborative filtering (PMF) and topic modeling (LDA) to use both ratings and documents.
\item Convolutional Matrix Factorization - state of the art movie prediction system that integrates a CNN with PMF. 
\subitem Kim, Donghyun. "Convolutional Matrix Factorization for Document Context-Aware Recommendation"
\end{itemize}


\end{document}